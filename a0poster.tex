\documentclass[25pt, a0paper, portrait]{tikzposter}
\usepackage{luatexja}
\usepackage{luatexja-fontspec}

\usetheme{Simple}
%\usetheme{Minimal}
%\usetheme{Basic}

%\useblockstyle{Minimal}
%\useblockstyle{Basic}
\useblockstyle{Slide}
%\useblockstyle{Default}

\title{\shortstack{ECS によるフルリゾルバのパフォーマンスに与える\\
影響の調査と解決策の提案・評価}}
\author{25GJK04 古賀 陽光}
\institute{下川研究室}

\makeatletter

\newcommand{\VILLAGE}{\@setfontsize\VILLAGE{45pt}{50pt}} % ブロックタイトル用 (非常に大きく)
\newcommand{\READABLE}{\@setfontsize\READABLE{43pt}{48pt}} % 本文用 (読みやすく)
% キャプションのフォントサイズを本文と同じにする
\renewcommand{\READABLE}{%
    \@setfontsize\READABLE{43pt}{48pt}%
    \let\normalsize\READABLE
}

\makeatother

% --- ここからブロックの再定義 ---
% 1. 既存の\blockコマンドを\oldblockとして保存
\let\oldblock\block 

% 2. 新しい\blockコマンドを定義し、自動的にサイズを適用する
% {<title>}{<text>}を引数にとり、旧コマンド(\oldblock)で内部的にサイズを強制します。
\renewcommand{\block}[2]{%
  \oldblock{\VILLAGE #1}{\READABLE #2}%
}
% --- ここまでブロックの再定義 ---

\begin{document}

\maketitle

% 自動番号付きブロック
\newcounter{blocknum}
\newcommand{\autoblock}[2]{%
  \stepcounter{blocknum}%
  \block{\theblocknum. #1}{#2}%
}
\begin{columns}
    \column{0.5} % 左カラム(幅50%)
    
    \autoblock{背景}{
        \begin{itemize}
            \item コンテンツの迅速かつ安定的な配信のため、コンテンツ配信サーバを地理的に分散配置する手法が広く採用
            \item この環境では、利用者ごとに最適なサーバへ接続を誘導する「リクエストナビゲーション」が重要
            \item 本研究では、DNS の名前解決時にこれを実現する方法を取り上げる
            \item DNS ベースのリクエストナビゲーションでは、一般的にクライアントの IP アドレスで最適なサーバを選択
            %\item 名前解決は、ドメイン名を IP アドレスに変換する技術
            %\item DNS の仕組み上、利用できるクライアントの情報は、名前解決の問い合わせ元の IP アドレスだけである
            %\item ただし、DNS の仕組み上、名前解決の問い合わせ元のIPアドレスはクライアントの IP アドレスではなく、フルリゾルバの IP アドレスである
            \item 従来は、この方法で最適なサーバの選択を実現
            %\item これは「フルリゾルバはクライアントとネットワーク的に近い」という前提があったため実現されていた
            %\item この前提が成り立っているため、クライアントの IP アドレスではなく、フルリゾルバの IP アドレスを用いて最適なサーバ選択が実現できる
            \item しかし、誰でも利用可能な高速な応答を特徴とする Public DNS の登場により、リクエストナビゲーションが出来ないケースが発生(図\the\numexpr\value{figure}+1\relax)
        \end{itemize}

        % 図を確実に中央寄せし、キャプションは画像の下で中央配置
        \begin{center}
                  \begin{minipage}{0.64\linewidth}
                    \centering
                    \includegraphics[width=\linewidth]{img/reqnavi.png}
                    \refstepcounter{figure}
                    {\normalsize 図\thefigure\quad 分散配置と誘導のイメージ}
                  \end{minipage}
                \end{center}

            }   
        \autoblock{ECS(EDNS Client Subnet)}{
            \begin{itemize}
                    \item DNS の名前解決要求にクライアントの IP アドレスの一部(サブネット情報)を付与する技術
                    \item Public DNS を利用している状況でも、最適なコンテンツ配信サーバの選択が可能
            \end{itemize} 
    }

    \autoblock{課題}{
         \begin{itemize}
            % 口頭説明する
            %\item フルリゾルバは、権威 DNS サーバからの応答をキャッシュすることで名前解決の高速化を図っている
            \item ECS 対応フルリゾルバでは、キャッシュエントリ数の増加によってキャッシュ効率が低下し、メモリ使用量や名前解決時間に悪影響を及ぼす問題
            \item 192.0.2.0, 192.0.3.0, 192.51.100.0, 192.51.101.0 というサブネット情報を使い 4回 example.com を名前解決した時、キャッシュエントリ数がフルリゾルバの設定によって増加する例(表\the\numexpr\value{table}+1\relax)
         \end{itemize}
        \begin{center}
            \begin{minipage}{0.9\linewidth}
                \centering
                \includegraphics[width=\linewidth]{img/cache-entry-list.png}
                \refstepcounter{table}
                {\normalsize 表\thetable\quad キャッシュエントリ数の例}
            \end{minipage}
        \end{center}
    }

    \autoblock{研究目的}{
        \begin{itemize}
            \item ECS 対応フルリゾルバのメモリ使用量と名前解決時間に与える影響がどのくらいかを調査する
            \item 課題の解決策を提案し、その効果を評価する
        \end{itemize}
    }

    \column{0.5} % 右カラム(幅50%)

    \autoblock{影響の調査}{
        % 子セクション番号(各ブロックごとにリセット)
        \newcounter{subsec}[blocknum]
        \newcommand{\subsec}[1]{%
          \refstepcounter{subsec}%
          \textbf{\theblocknum.\arabic{subsec}. #1}\par
        }

        \subsec{概要}
        \begin{itemize}
            \item ECS によってメモリ使用量、名前解決時間がどのくらい変わるかを5回実験して平均値を調査
        \end{itemize}

        \subsec{環境}
        \begin{itemize}
            \item ECS 対応フルリゾルバを構築し、クライアントから調査用フルリゾルバに名前解決を要求(図\the\numexpr\value{figure}+1\relax)
            \item フルリゾルバのソフトウェアは Unbound を使用
            \item 調査用フルリゾルバは、フルリゾルバ別に設定したサブネット情報の長さ(/16, /22, /24)に丸めて名前解決
            \item 5,000件の実在するドメイン名と8種類のサブネット情報を組み合わせて名前解決
            \item サブネット情報は133.17.1.0/32から133.17.8.0/32の8通りを使用
        \end{itemize}
        \begin{center}
            \begin{minipage}{0.6\linewidth}
                \centering
                \includegraphics[width=\linewidth]{img/dns-env.jpg}
                \refstepcounter{figure}
                {\normalsize 図\thefigure\quad 実験環境}
            \end{minipage}
        \end{center}

        \subsec{調査結果}
        \begin{itemize}
            \item 図\the\numexpr\value{figure}+1\relax はメモリ使用量推移である。丸めるサブネット情報が短い(/16)ほどメモリ使用量は小さく、長い(/24)ほど大きい傾向
            \item 図\the\numexpr\value{figure}+2\relax の平均名前解決時間は、ECS を利用するフルリゾルバは ECS 無しと比べて50ミリ秒程度名前解決時間が増加
        \end{itemize}
        \begin{center}
            \begin{minipage}{0.75\linewidth}
                \centering
                \includegraphics[width=\linewidth]{img/renzoku-subnet-graph.png}
                \refstepcounter{figure}
                {\normalsize 図\thefigure\quad メモリ使用量推移}
            \end{minipage}\hfill
        \end{center}
        \begin{center}
            \begin{minipage}{0.6\linewidth}
                \centering
                \includegraphics[width=\linewidth]{img/ecs-resolv-time-big.png}
                \refstepcounter{figure}
                {\normalsize 図\thefigure\quad 平均名前解決時間}
            \end{minipage}
        \end{center}

    }   
    \autoblock{今後の展望}{
        \begin{itemize}
            \item 今回フルリゾルバのソフトウェアは Unbound を使用
            \begin{itemize}
                \item 今後 PowerDNS Recursor でも実験
            \end{itemize} 
            \item 調査結果を踏まえ、メモリ使用量と名前解決時間を軽減するための解決策を提案し、その効果を評価
            \begin{itemize}
                \item ECS 有効ドメインと無効ドメインを判別して名前解決する方法を今後実験する予定
            \end{itemize}
        \end{itemize}
    }
    
\end{columns}

\end{document}
